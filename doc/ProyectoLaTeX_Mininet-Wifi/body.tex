\chapter{Mininet-Wifi}

Por lo que he leído Mininet-Wifi no incorpora el estándar que usábamos para el estudio de las redes de área personal con tasa baja de transmisión de datos, 802.15.4 .
\section{Enlaces útiles}
\begin{itemize}
    \item Articulos sobre Mininet y Mininet-Wifi: \url{http://www.brianlinkletter.com/tag/mininet/}
    
\end{itemize}


\section{¿Qué es?}


Mininet-Wifi es un fork del proyecto Mininet con el que podías emular redes SDN, y se han extendido sus funcionalidades al ámbito de las redes wireless. Por lo que con el se pueden combinar ambas funcionalidades, las anteriores comunes a Mininet y las nuevas que incorpora Mininet-Wifi.

\section{¿Qué es el término RSSI?}

El indicador de fuerza de la señal recibida (RSSI por las siglas del inglés Received Signal Strength Indicator), es una escala de referencia \textbf{en relación a 1 mW} para medir el nivel de potencia de las señales recibidas por un dispositivo en las redes inalámbricas (típicamente WIFI o telefonía móvil).



% Please add the following required packages to your document preamble:
% \usepackage{booktabs}

\begin{table}[!htb]
\centering
\begin{tabular}{@{}ll@{}}
\toprule
RSSI & Descripción                                                                                                                                                        \\ \midrule
0    & Señal ideal                                                                                                                                                        \\
-40  & Señal idónea con tasas de transferencia estables.                                                                                                                  \\
-60  & \begin{tabular}[c]{@{}l@{}}Enlace bueno, ajustando la \\ transmisión (Tx) se puede lograr una conexión estable al 80\%\end{tabular}                                \\
-70  & \begin{tabular}[c]{@{}l@{}}Enlace medio-bajo, es una señal medianamente buena\\ aunque se pueden sufrir problemas con lluvia y viento.\end{tabular}                \\
-80  & \begin{tabular}[c]{@{}l@{}}Es la señal mínima aceptable para establecer la conexión.\\ Pueden ocurrir caídas que se traducen en corte de comunicación\end{tabular} \\ \bottomrule
\end{tabular}
\centering 
\end{table}

\newpage

\section{Instalación}

Instalar Mininet-Wifi me ha resultado muy sencillo. Se lleva a cabo vía un shellscript que te dan ya hecho al clonar el repositorio de Mininet-Wifi. Los pasos que he seguido en la instalación de Mininet-Wifi son:
\begin{itemize}
    \item Instalar una máquina virtual con Ubuntu 16.04
    \item Añadir git, sudo apt-get install git 
    \item Clonar el repositorio, git clone https://github.com/intrig-unicamp/mininet-wifi
    \item cd mininet-wifi
    \item Completar la instalación: sudo \textbf{util/install.sh -Wlnfv}
    \begin{itemize}
        \item -W: wireless dependencies
        \item -n: mininet-wifi dependencies
        \item -f: OpenFlow
        \item -v: OpenvSwitch
        \item -l: wmediumd
    \end{itemize}
    \item De forma adicional hemos instalado Wireshark para poder analizar los test realizados
\end{itemize}

\section{Test 1}
La red más sencilla es por defecto la red de un punto de acceso y dos estaciones wireless. El punto de acceso está conectado directamente al controlador, y las estaciones wireless serán los host. La idea de este test es comprobar que hay fluctuación de tráfico de control OpenFlow en el punto de acceso y además hay tráfico de usuario en la interfaz de wlan.

\begin{itemize}
    \item Lanzamos Wireshark para analizar tráfico: \textbf{wireshark \&}
    \item Lanzamos Mininet-Wifi con el siguiente comando, además por defecto cargará la topología por defecto(Valga la redundancia): \textbf{sudo mn -wifi}
    \item Lo siguiente, es activar la interfaz hwsim0. Pero, ¿Qué es la interfaz hwsim0? La interfaz hwsim0 es una interfaz software creada por Mininet-Wifi que copia todo el trafico wireless  dirigido a todas las interfaces wireless virtuales de la topología a emular. Según la documentación seguida es la forma más sencilla de monitorizar los mensajes wireless en Mininet-Wifi. Desde la Mininet-Wifi CLI: \textbf{sh ifconfig hwsim0 up} 
    \begin{itemize}
        \item El comando \textbf{sh} en la CLI de Mininet tiene la funcionalidad de ejecutar un comando fuera de la Interfaz de Mininet-Wifi.
    \end{itemize}
    
    \item Una vez levantada la interfaz podemos poner Wireshark a escuchar en la interfaz.
    
    \item Hacemos Ping desde la estación sta1, a la estación sta2: \textbf{sta1 ping sta2}
    
    \item Comprobamos que hay tráfico escuchando en la interfaz:
    
    
\end{itemize}
\newpage
\begin{figure}[!htb]
  \centering
    \includegraphics[width=\linewidth]{./img/5.JPG}
    \caption{Comprobación que hay trafico a través de la interfaz.}
  \label{fig:yo}
\end{figure}

Como podeos ver el punto de acceso tiene una interfaz asociada llamada ap1-wlan1. Por defecto, las estaciones wireless asociadas con el punto de acceso se conectan en modo "\textbf{infrastructure} " esto significa que el tráfico wireless entre dos estaciones asociadas al punto de acceso debe pasar siempre a través de este. Sabiendo que los puntos de acceso funcionan de forma similar a los switch en Mininet, esperaríamos observar tráfico de control entre el punto de acceso y el controlador, cuando el punto de acceso,  observe tráfico para el cual no se establecido una regla (No pertenece a un flujo para el cual haya asignada una acción en la tabla de flujos). 

\begin{figure}[!htb]
  \centering
    \includegraphics[width=0.9\linewidth]{./img/6.JPG}
    \caption{Interfaz ap1-wlan1 del Punto de acceso.}
  \label{fig:yo}
\end{figure}
\newpage
Si deseamos ver \textbf{paquetes OpenFlow}, debemos poner a escuchar Wireshark  en la interfaz de \textbf{loopback}. Podemos además utilizar el filtro: \textbf{OpenFlow\_1.0} para visualizarlo de una forma más clara. Dicho esto, ponemos a capturar Wireshark y repetimos el ping entre sta1 y sta2.
\begin{figure}[!htb]
  \centering
    \includegraphics[width=\linewidth]{./img/7.JPG}
    \caption{Tráfico OpenFlow.}
  \label{fig:yo}
\end{figure}
\newline
Nota: 
\begin{itemize}
    \item Consultar comando dpctl: \url{http://ranosgrant.cocolog-nifty.com/openflow/dpctl.8.html}
    \item xID: Identificador de transacción, es aleatorio, identificador controlador - switch, va cambiando a lo largo del tiempo.  
    \item El estándar es OpenFlow 1.3, pero Mininet-Wifi funciona con OpenFlow 1.1
\end{itemize}

Como esperábamos el paquete ICMP debería ser redirigido al controlador para decidir que hacer con el paquete, al no tener un Flow con una regla establecida. De esta manera el controlador cuando  le llegue el paquete instanciará una serie de reglas en los switch para que el paquete ICMP sea encaminado de un host a otro. En cambio, encontramos que las dos estaciones son capaces de intercambiar paquetes inmediatamente sin redirigir el primer paquete al controlador. Solo una trama ARP , que es de tipo broadcast, es llevada hacia el controlador y es ignorada.\newline
\newline
Para comprobar si hubira algún flujo con una ccción instanciada en alguna de las estaciones bases hacemos:
\textbf{dpctl dump-flows}
\newline
\newline
Al hacer esto en esta altura del tutorial, vemos que no hay ninguna regla instanciada en la estación. ¿Pero, tiene esto sentido?¿Cómo se han comunicado las dos puntos de acceso?\newline
\newline
Según la guía, podemos apreciar que los switches con OpenFlow activado, van a rechazar las "Hairpin connections". Las cuales son flows que hacen que el tráfico, sea enviado por el mismo puerto que ha sido recibido.

\begin{itemize}
    \item \textbf{Hairpin connection}: Comunicación entre dos dispositivos host en el mismo dominio NAT utilizando sus puertos finales(Ya traducidos)
\end{itemize}
\newpage
\begin{figure}[!htb]
  \centering
    \includegraphics[width=\linewidth]{./img/8.JPG}
    \caption{No hay tráfico OpenFlow hacia el controlador.}
  \label{fig:yo}
\end{figure}
Un punto de acceso wireless, por diseño, recibe y manda paquetes en la misma interfaz. Las estaciones wifi conectadas al mismo punto de acceso requerirán a "Hairpin connection" para comunicarse entre ellas. El autor de la guía supone que Linux, maneja a la interfaz WLAN en cada punto de acceso como un enlace radio sta1-ap1-sta2 como si fuera un HUB, donde ap1-wlan0 proporciona la funcionalidad de "HUB"
para los datos que pasan entre sta1 y sta2.
ap1-wlan0 cambia los paquetes en el dominio inalámbrico y lo hará
no introduzca un paquete en la parte del "conmutador Ethernet" del punto de acceso ap1 a menos que deba cambiarse a
otra interfaz en ap1 que no sea ap1-wlan0.\newline
\newline
\begin{itemize}
    \item Podemos detener el ping con CTRL+C
    \item Podemos detener Mininet-Wifi vía comando \textbf{exit}
    \item Podemos limpiar los archivos residuales de Mininet-Wifi con:  \textbf{sudo mn -c}
\end{itemize}

\section{Test 2}
En este test vamos a crear una topología lineal con tres puntos de acceso, y tres estaciones wifi. Donde cada una está conectada a cada punto de acceso.\newline
\newline
Podemos crear la topología: \textbf{sudo mn --wifi --topo linear,3}
\newpage
\begin{figure}[!htb]
  \centering
    \includegraphics[width=\linewidth]{./img/9.JPG}
    \caption{Montar topología lineal en Mininet-Wifi.}
  \label{fig:yo}
\end{figure}
Además,  podemos  verificar la configuración de la topología haciendo uso de los comandos \textbf{net} y \textbf{dump}. Con el comando \textbf{net} podemos ver la conexión entre los nodos. Con el comando \textbf{dump} podemos ver la conexión entre los nodos y además información extra como el PID.\newline
\newline
Podemos contemplar que tenemos los tres puntos de acceso conectados juntos entre ellos uno a uno de una manera lineal mediante enlaces de Ethernet. Pero, no vemos ningún tipo de información sobre las estaciones Wifi, en lo relativo a que punto de acceso están conectadas. Esto se debe a que esa conexión se lleva a cabo mediante un enlace te tipo radio. Para poder consultar esta información tendremos que hacer uso del comando \textbf{iw} (Interface Wireless) en cada estación Wifi para poder contemplar a que punto de acceso está conectada. \newline
\newline
Para poder verificar que puntos de acceso son alcanzables por una estación Wifi usaremos \textbf{staX iw dev staX-wlan0 scan}. Para afinar la busqueda podemos hacer un grep del ssid para quitarnos información extra.\newline
\newline
\begin{center}
    \textbf{sta1 iw dev sta1-wlan0 scan | grep ssid}
\end{center}
\begin{figure}[!htb]
  \centering
    \includegraphics[width=0.5\linewidth]{./img/10.JPG}
    \caption{Comprobar que puntos de accesos son alcanzables en Mininet-Wifi.}
  \label{fig:yo}
\end{figure}
Para comprobar a que punto de acceso está conectado en un momento dado una estación Wifi podemos hacerlo mediante el comando \textbf{iw link}. Por ejemplo: \textbf{sta1 iw dev sta1-wlan0 link
}
\newpage
\begin{figure}[!htb]
  \centering
    \includegraphics[width=0.5\linewidth]{./img/11.JPG}
    \caption{Comprobar el punto de acceso actual en Mininet-Wifi.}
  \label{fig:yo}
\end{figure}
Con este comando  podemos apreciar distintas estadísticas como pueden ser la potencia de señal recibida, la cantidad de bytes transmitidos y recibidos, el régimen binario de transmisión, SSID, y la frecuencia expresada en MHz, en este caso se trata del canal 1 Wifi. Debemos recordar que el espectro Wifi se extiende desde 2402 MHz hasta los 2494 Mhz (La banda de los 2.4GHz).\newline
\newline
En este caso cada estación Wifi está conectad a un punto de acceso según se pudo ver con los anteriores comandos de \textbf{net} y \textbf{dump}. Se puede hacer uso del comando \textbf{iw} para conmutar de un punto de acceso por otro. \newline
\newline
Nota de la guia: \textit{Los comandos iw pueden usarse en escenarios estáticos como este, pero no deben usarse cuando
Mininet-WiFi asigna automáticamente asociaciones en escenarios de movilidad más realistas. Discutiremos
cómo Mininet-WiFi maneja la movilidad real y cómo usar los comandos iw con Mininet-WiFi más adelante
esta publicación}.\newline
\newline
En este caso vamos a comprobar como podemos cambiar la conexión existente entre la estación Wifi 1 (sta1) con el punto de acceso 1 (ap1). En este caso vamos a conectar la estación Wifi 1 (sta1) con el punto de acceso 2 (ap2). Debemos recordar los SSID de cada punto de acceso (ssid\_apX).\newline
\newline
\begin{center}
\begin{itemize}
    \item  Para desconectarnos del actual punto de acceso:  \textbf{sta1 iw dev sta1-wlan0 disconnect}
    \item Para conectarnos a ap2: \textbf{sta1 iw dev sta1-wlan0 connect ssid\_ap2}
\end{itemize}
\end{center}
Podemos verificar el funcionamiento de este comando usando el comando iw link que ya se explico anteriormente.
\begin{figure}[!htb]
  \centering
    \includegraphics[width=0.5\linewidth]{./img/12.JPG}
    \caption{\textit{Handover} completado en Mininet-Wifi.}
  \label{fig:yo}
\end{figure}
Hemos podido comprobar como se puede hacer un simple \textit{Handover} entre dos puntos de acceso.
\newpage
Ahora vamos a estudiar como gestiona el controlador OVSAP de Mininet-Wifi el Handover entre dos células Wifi. Para llevar a cabo este experimento vamos hacer ping desde sta3 hacia sta1. Entraremos en la terminal de sta3 con el comando \textbf{xterm}. Para hacer ping debemos saber las distintas IPs de las distintas estaciones Wifi, podemos consultarlas con el comando \textbf{dump}.
\begin{figure}[!htb]
  \centering
    \includegraphics[width=\linewidth]{./img/13.JPG}
    \caption{Consultar IPs en Mininet-Wifi.}
  \label{fig:yo}
\end{figure}
\begin{figure}[!htb]
  \centering
    \includegraphics[width=\linewidth]{./img/14.JPG}
    \caption{Entrar en la consola de \textit{Sta3}.}
  \label{fig:yo}
\end{figure}
\newline
Vamos hacer ping desde sta3 a sta1 (\textbf{Ip: 10.0.0.1}). Ahora desde que estos paquetes serán reenviados por los puntos de acceso asociados, en un puerto distinto al puerto
en el que se recibieron los paquetes, los puntos de acceso funcionarán normalmente en cuanto a OpenFlow ser refiere. Cada punto de acceso va a reenviar el primer ping que le llegue en dirección al controlador. El controlador va a instanciar en las tablas de flows de cada punto de acceso las reglas correspondientes para que se pueda instanciar una conexión entre sta1 y sta3. \newline
\newline
Podemos comprobar el intercambio de paquetes OpenFlow escuchando la interfaz de loopback.
\newpage
\begin{figure}[!htb]
  \centering
    \includegraphics[width=0.8\linewidth]{./img/15.JPG}
    \caption{Ping desde \textit{Sta3} a \textit{Sta1}.}
  \label{fig:yo}
\end{figure}
\begin{figure}[!htb]
  \centering
    \includegraphics[width=\linewidth]{./img/16.JPG}
    \caption{Captura de Wireshark del ping.}
  \label{fig:yo}
\end{figure}
\newpage
Podemos apreciar como el punto de acceso ap3 empieza mandando el paquete de ARP request para que Sta3 sepa la \textit{MAC} de Sta1 hacia arriba para que decida que debe hacer con el controlador. El controlador hace un PACKET\_OUT. Para el ARP reply el controlador hace un PACKET\_IN. Una vez que Sta3 sabe sabe la dirección \textit{MAC} del destino se dispone a enviar el ping. El controlador instanciará un flow\_mod por cada punto de acceso, y otro para el ECHO reply del ping.

\begin{figure}[!htb]
  \centering
    \includegraphics[width=\linewidth]{./img/17.JPG}
    \caption{Flow tables de los distintos puntos de acceso.}
  \label{fig:yo}
\end{figure}
\newline
Como se puede apreciar en la figura, al principio de la comprobación de las tablas de cada punto de acceso estaban vacias. Esto es ya que estas entradas en la tabla tienen un tiempo de vida, agotado se eliminan de la tabla. Por lo que al repetir el ping se vuelve a repetir el proceso de instanciar los Flow\_Mod en cada tabla de cada punto de acceso para poder llevarse así a cabo el ping. \newline
\newline

Pero resulta curioso el hecho de que en le punto de acceso 1 (ap1) no haya ninguna entrada en su tabla de Flows. Esto es ya que este punto de acceso no interviene ya que anteriormente hicimos un Handover de Sta1, entre ap1 y el ap2. Podemos comprobar que si volvemos a enlazar Sta1 con ap1, ap1 deberá tener también las entradas suficientes en su Flow table para poder llevar a cabo la conexión entre Sta1 y Sta3. 
\begin{itemize}
    \item \textbf{sta1 iw dev sta1-wlan0 disconnect} 
    \item \textbf{sta1 iw dev sta1-wlan0 connect ssid\_ap1}
\end{itemize}
\begin{figure}[!htb]
  \centering
    \includegraphics[width=0.9\linewidth]{./img/18.JPG}
    \caption{Flow tables de los distintos puntos de acceso.}
  \label{fig:yo}
\end{figure}
\newpage
Si este proceso lo hacemos lo suficientemente rápido nos daremos cuenta que al mover Sta1 el ping no funciona. Esto se debe a que las entradas de la tabla de Flows siguen configuradas para llevar un paquete desde ap3 hasta ap2, pero no a ap1. Si pasa el tiempo suficiente estas entradas serán desechadas o bien, podemos quitarlas nosotros para que se creen unas entradas en las tablas de Flows nuevas y que así se pueda alcanzar Sta1 desde su nueva posición.
\begin{itemize}
    \item \textbf{dpctl del-flows}
\end{itemize}
Llegados a este punto, ya habríamos acabado el test 2, por lo que solo nos quedaría salir y limpiar Mininet-Wifi. Como ya se explicó:

\begin{itemize}
    \item (Desde la CLI de Mininet-Wifi) exit
    \item  sudo mn -c
\end{itemize}
\newpage
\section{Test 3}

Mininet proporciona una API de Python para que los usuarios puedan crear secuencias de comandos de Python simples que configurarán 
topologías personalizadas en Mininet-WiFi. Mininet-WiFi extiende esta API para soportar un entorno inalámbrico a la ya existente API de Mininet para Python.\newline
\newline
\begin{figure}[!htb]
  \centering
    \includegraphics[width=0.8\linewidth]{./img/mininet-wifi-logo.png}
\end{figure}

Cuando se usa el comando normal Mininet \textbf{mn} con la opción –-wifi para crear Mininet-WiFi, no se tiene acceso a la mayoría de las funciones ampliadas que se proporcionan en Mininet-WiFi para crear topologías. Para acceder a las funciones que le permiten emular el comportamiento de los nodos en una LAN inalámbrica, se debe
usar las extensiones de Mininet-Wifi para la API de Mininet de Python.\newline
\newline
Las grandes diferencias que tiene la nueva API de Mininet-Wifi respecto a la anterior de Mininet, es que han añadido nuevos métodos al objeto de la topología, llamados \textit{addStation} y \textit{addAccessPoint}. Además han añadido código y han modificado gran parte de él para añadir el factor wireless a los enlaces, en el método \textit{addLink}. La guía nos aconseja empezando a mirar los ejemplos de scripts ya hechos en Python.\newline

\subsection{Métodos básicos API Mininet-Wifi}
\newline
A continuación, se va a exponer los métodos más básicos para trabajar con Mininet-Wifi. En los escenarios anteriores al ser escenarios por defecto se pudo apreciar que al no suministrar valores a los parámetros de los elementos de la red, se aplicaron los valores por defecto. Ahora desde los scripts creados se podrá especificar con exactitud los valores que queremos. 

% Para los scripts 
%\begin{minted}[xleftmargin=5pt,linenos,fontsize=\small]{python}
\begin{minted}[]{python}
#De esta manera añadiríamos una estación Wifi por defecto

net.addStation( 'sta1' )


#Con esta sentencia añadiriamos un nuevo punto de acceso,
# pero a diferencia de los punto de acceso usados, 
# este no tendrá un SSID por defecto apX-ssid, si no, el suministrado.

net.addAccessPoint( 'ap1', ssid='new_ssid' )


#Para añadir un enlace Wireless entre ambos se hace con la siguiente
# sentencia, pero el enlace tendrá los valores por defecto.

net.addLink( ap1, sta1 )

\end{minted}

Para crear escenarios más complicados y complejos se puede explotar todos los posibles parámetros que recogen estos métodos. Entre los cuales, se puede indica, MAC, IP, localización (x,y,z), radio de alcance, y mucho más. Por ejemplo, el siguiente código define un punto de acceso y una estación, y crea una asociación (una conexión inalámbrica) entre los dos nodos y aplica algunos parámetros de control de tráfico a la conexión a de radio realista, agregando restricciones de ancho de banda, una tasa de error de bit y un
retardo de propagación.

\begin{minted}[]{python}

# Añadimos una estación Wifi con un método de encriptado,
# MAC, Ip, posición, contraseña y nombre.

net.addStation( 'sta1', passwd='123456789a', encrypt='wpa2',
 mac='00:00:00:00:00:02', ip='10.0.0.2/8', position='50,30,0' )
 
# Añadimos un punto de acceso indicándole, método de encriptado,
# MAC, SSID, canal, modo de wifi
\end{minted}

