\section{Modelos de Propagación en Mininet-Wifi}
\subsection{Tipos de modelos}
Mininet-WiFi implementa los siguientes modelos de propagación:
\begin{itemize}
    \item Modelo de pérdidas de propagación de FRIIS
    \item Modelo de pérdidas de propagación de Distancia (Por defecto)
    \item Modelo de pérdidas Log-normal path
    \item Modelo de pérdidas de propagación de la Unión Internacional de Telecomunicaciones (UIT)
    \item Modelo de pérdidas de propagación de 2 rayos 
\end{itemize}
Podemos elegir entre estos modelos de propagación via API de Python de Mininet-Wifi. La topología tiene un método llamado \textit{net.propagationModel()} . Ejemplos:

\begin{minted}[]{python}
#Ejemplos extraídos del manual

# Friis Propagation Loss Model
net.setPropagationModel(model="friis", sL)

sL = system loss (int)

# Log-Distance Propagation Loss Model
net.setPropagationModel(model="logDistance", sL , exp )

sL = system loss (int)
exp = exponent   (int)

#Log-Normal Shadowing Propagation Loss Model
net.setPropagationModel(model="logNormalShadowing", sL , exp ,
, variance )

sL = system loss (int)
exp = exponent (int)
variance = gaussian variance (int)

# International Telecommunication Union (ITU) Propagation Loss Model
net.setPropagationModel(model="ITU", lF , nFloors , pL )

lF = floor penetration loss factor (int)
nFloors = number of floors (int)
pL = power loss coefficient (int)


# Two-Ray Ground Propagation Loss Model
# Nota: Este modelo no da benos resultados en la corta distancia.
net.setPropagationModel(model="twoRayGround")

\end{minted}
\newpage